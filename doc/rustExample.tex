\IfFileExists{t2doc.cls}{
    \documentclass[documentation]{subfiles}
}{
    \errmessage{Error: could not find 't2doc.cls'}
}

\begin{document}

\trantitle
    {rustExample} % Plugin name
    {T2 Rust plugin example} % Short description
    {Tranalyzer Development Team} % author(s)

\section{rustExample}\label{s:rustExample}

\subsection{Description}
This plugin is an example on how to use the {\tt t2plugin} crate to create a Tranalyzer2 plugin
in Rust.

\begin{description}
    \item[{\tt t2plugin} crate source:] {\url{https://github.com/Tranalyzer/t2plugin}}
    \item[{\tt t2plugin} crate documentation:] {\url{https://tranalyzer.com/rustdoc/t2plugin/}}
\end{description}

This plugin performs the following three tasks for each flow:

\begin{enumerate}
    \item Compute the on-wire throughput (from layer 2). This demonstrates how to output a simple
        column and how to access the {\tt Packet} and {\tt Flow} structures.
    \item Extract the {\tt PHPSESSID} cookies from HTTP. This demonstrates how to output a
        compound column and how to parse text protocols.
    \item Extract the Server Name Indication (SNI) from TLS handshakes. This demonstrates how to
        parse binary protocols.
\end{enumerate}

\subsection{Flow File Output}
The rustExample plugin outputs the following columns:
\begin{longtable}{llll}
    {\bf Column} & {\bf Type} & {\bf Description}\\
    \hline\endhead
    {\tt l2Throughput} & D & On-wire throughput in [bytes/s], computed from layer 2.\\
    {\tt phpSessIds} & R:U8\_S & Repetitive compound: \nameref{phpSessId}.\\
    {\tt tlsSni} & S & TLS handshake Server Name Indication (SNI) extension.\\
\end{longtable}

\subsubsection{phpSessId}\label{phpSessId}
Each compound value in the {\tt phpSessIds} column is to be interpreted as follows:

\begin{longtable}{rl}
    {\bf 1st sub-value} & {\bf Description}\\
    \hline\endhead
    {\bf 0} & Cookie sent by the client in a {\tt Cookie} header\\
    {\bf 1} & Cookie sent by the server in a {\tt Set-Cookie} header.\\
\end{longtable}

The 2nd sub-value contains the value of the {\tt PHPSESSID} cookie.

\end{document}
